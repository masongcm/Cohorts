
%%%%%%%%%%%%%%%%%%%%%%%%%%%%%%%%%%%%%%%%%%%%%%%%%%%%%%%%%%%%%%%%%%%%%%%%%%%%%%%%%%%%%%%%%

\begin{table}[ht!]
\caption{Behavioural screening scales in the BCS and MCS five-year surveys}\label{tab:scales}
\centering\setstretch{1.2} \footnotesize
\begin{tabular}{L{.46\linewidth}L{.001\linewidth}L{.46\linewidth}}
\toprule
\multicolumn{3}{l}{\textbf{Panel A: Rutter A Scale \citep{Rutter1970}} -- British Cohort Study (1975) five-year survey} \\[-1.8em]
\begin{enumerate}
\item Very restless. Often running about or jumping up and down. Hardly ever still.$^\dagger$
\item Is squirmy or fidgety.$^\dagger$
\item Often destroys own or others' belongings.
\item Frequently fights other children.$^\dagger$
\item Not much liked by other children.
\item Often worried, worries about many things.$^\dagger$
\item Tends to do things on his/her own, is rather solitary.$^\dagger$
\item Irritable. Is quick to fly off the handle.
\item Often appears miserable, unhappy, tearful or distressed.$^\dagger$
\item Sometimes takes things belonging to others.
\item Has twitches, mannerisms or tics of the face or body.
\item Frequently sucks thumb or finger.
\end{enumerate}
& &
\begin{enumerate}\setcounter{enumi}{12}
\item Frequently bites nails or fingers.
\item Is often disobedient.$^\dagger$
\item Cannot settle to anything for more than a few moments.$^\dagger$
\item Tends to be fearful or afraid of new things or new situations.$^\dagger$
\item Is over fussy or over particular.
\item Often tells lies.
\item Bullies other children.$^\dagger$
\item[A.] Complains of headaches.$^\dagger$
\item[B.] Complains of stomach-ache or has vomited.$^\dagger$
\item[C.] Complains of biliousness.
\item[D.] Has temper tantrums (that is, complete loss of temper with shouting, angry movements, etc.).$^\dagger$
\end{enumerate}
\\
\midrule
\multicolumn{3}{l}{\textbf{Panel B: Strength and Difficulties Questionnaire \citep{Goodman1997}} -- Millennium Cohort Study (2000/1) five-year survey} \\[-1.8em]
\begin{enumerate}
\item Considerate of other people's feelings.
\item Restless, overactive, cannot stay still for long.$^\dagger$
\item Often complains of head- aches, stomach-ache or sickness.$^\dagger$
\item Shares readily with other children (treats, toys, pencils, etc.).$^+$
\item Often has temper tantrums or hot tempers.$^\dagger$
\item Rather solitary, tends to play alone.$^\dagger$
\item Generally obedient, usually does what adults request.$^{\dagger +}$
\item Many worries, often seems worried.$^\dagger$
\item Helpful if someone is hurt, upset or feeling ill.$^+$
\item Constantly fidgeting or squirming.$^\dagger$
\item Has at least one good friend.$^+$
\item Often fights with other children or bullies them.$^\dagger$
\item Often unhappy, down-hearted or tearful.$^\dagger$
\end{enumerate}
& &
\begin{enumerate}\setcounter{enumi}{13}
\item Generally liked by other children.$^+$
\item Easily distracted, concentration wanders.$^\dagger$
\item Nervous or clingy in new situations, easily loses confidence.$^\dagger$
\item Kind to younger children.$^+$
\item Often lies or cheats.
\item Picked on or bullied by other children.
\item Often volunteers to help others (parents, teachers, other children).$^+$
\item Thinks things out before acting.$^+$
\item Steals from home, school or elsewhere.
\item Gets on better with adults than with other children.
\item Many fears, easily scared.
\item Sees tasks through to the end, good attention span.$^+$
\end{enumerate}
\\
\bottomrule
\end{tabular}
\justify {\scriptsize \emph{Notes}: Items denoted by $^+$ are positively coded in the original scale. Items denoted by $^\dagger$ are retained in the new comparable scale.
\par}
\end{table}

%%%%%%%%%%%%%%%%%%%%%%%%%%%%%%%%%%%%%%%%%%%%%%%%%%%%%%%%%%%%%%%%%%%%%%%%%%%%%%%%%%%%%%%%%
\begin{landscape}
\begin{table}[ht!]
\caption{Description of harmonised variables}\label{tab:harmvar}
\centering\setstretch{1.2} \footnotesize
\begin{tabular}{L{.2\textwidth}C{.1\textwidth}L{.3\textwidth}L{.7\textwidth}}
\toprule
Variable Group & Age & Variable & Note \\
\midrule
Maternal education & 5 & Post-compulsory schooling$^d$ & Whether mother continued schooling past the compulsory age, based on her year of birth. School leaving age in England was changed from 14 to 15 in 1947 and from 15 to 16 in 1972. \\ \\[-1.5em]
Maternal employment & 5 & Employed$^d$ & Includes full time and part time \\ \\[-1.5em]
Father occupation & 5 & White collar (I-IIINM)$^d$ \newline Blue collar (IIIM-V-other)$^d$ \newline No father figure$^d$ & Based on father's Registrar General Social Class classification of occupations. White collar includes I (Professional), II (Managerial/technical), IIINM (Skilled non-manual). Blue collar includes IIIM (Skilled manual), IV (Partly skilled), V (Unskilled), Other, Unemployed, and Armed forces. No father figure is a dummy for children whose father does not live in the same household. \\ \\[-1.5em]
Maternal background & 0/5 & Mother's age at birth \newline Mother's height (cm) \newline Mother unmarried at birth$^d$ \newline Child nonwhite ethnicity$^d$  \newline Number of children in HH & All variables are self-reported by the mother at birth, except for number of children in household (at five years old). Unmarried is only based on marital status, and includes cohabitation. \\ \\[-1.5em]
Pregnancy & 0 & Child is firstborn$^d$ \newline Number of previous stillbirths \newline Mother smoked in pregnancy$^d$ \newline Preterm birth (under 37 weeks gestation)$^d$ \newline (log) birth weight (kg) & Parity, stillbirths, and smoking are self-reported by the mother. Gestational length and birth weight are from hospital records. \\ \\[-1.5em]
Cognitive skills & 5 & & Based on test batteries administered to the cohort member at five. Three tests are used for BCS children: Copy Designs child is asked to copy simple designs adjacently), Human Figure Drawing (child draws an entire human figure), English Picture Vocabulary Test (child identifies the picture referring to a word among four pictures). Three \emph{different} tests are used in the MCS: BAS Naming Vocabulary (child is shown a series of pictures and asked to name it), BAS Picture Similarity (child is shown a row of 4 pictures on a page and places a card with a fifth picture under the one most similar to it), BAS Pattern Construction (child constructs a design by putting together flat squares or solid cubes with patterns on each side).

 \\ \\[-1.5em]
Adolescent outcomes & 16 (BCS) \newline 14 (MCS) & Child tried smoking$^d$ \newline Body Mass Index (BMI) & Smoking is self reported by the child. Height and weight are taken as part of a medical examination. \\ \\[-1.5em]
Adult outcomes (BCS only) & 34 \newline 42 \newline 34, 42 & Higher education$^d$ \newline Employed$^d$ \newline (log) gross weekly pay & Higher education is defined on having a university degree or its vocational equivalent. It corresponds to level 4 or 5 in the National Vocational Qualification (NVQ) equivalence. Employed is a dummy for being in paid employment or self-employment, either full or part time. Gross weekly pay is weekly pre-tax pay from the respondent's main activity, conditional on being a paid employee. \\
\bottomrule
\end{tabular}
\justify {\scriptsize \emph{Notes}: Variables denoted by $^d$ are binary or categorical.
\par}
\end{table}
\end{landscape}

%%%%%%%%%%%%%%%%%%%%%%%%%%%%%%%%%%%%%%%%%%%%%%%%%%%%%%%%%%%%%%%%%%%%%%%%%%%%%%%%%%%%%%%%%

\begin{table}[ht]
\centering\setstretch{1.2} \footnotesize
\caption{Parameterisations for measurement invariance}\label{tab:invparam}
\begin{tabular}{L{.18\textwidth}L{.45\textwidth}L{.3\textwidth}}
\toprule
\textbf{Invariance level}     & \textbf{Description} & \textbf{Restrictions} \\ \midrule \\
Configural (WE{\textTheta})   
    & \bi[label=$\cdot$] \item Minimally restrictive model for identification \ei
    & For all groups: $\begin{array}{|l} \text{diag}(\bm{\Phi}) = \bm{I} \\
                      \bm{\kappa}=\bm{0} \\
                      \bm{\nu}=\bm{0} \\
                      \text{diag}(\bm{\Psi})=\bm{I} \end{array}$
                      \\ \\
Threshold invariance
    & \bi[label=$\cdot$] \item Restricts thresholds $\tau$ to be equal across groups
                  \item Statistically equivalent to configural (when measures have 3 categories or less)\ei
    & $\tau_{1,ci} = \tau_{1,c^\prime i}$ for all items, $\forall c, c^\prime$
                      \newline $\tau_{2,ci} = \tau_{2,c^\prime i}$ for non-binary items, $\forall c, c^\prime$
                      \newline For all groups: $\begin{array}{|l} \text{diag}(\bm{\Phi}) = \bm{I} \\ 
                                                \bm{\kappa}=\bm{0} \end{array}$
                      \newline For ref. group $A$: $\;\begin{array}{|l} \bm{\nu}_A=\bm{0} \\
                                                  \text{diag}(\bm{\Sigma}_A)=\bm{I} \end{array}$
                   \\ \\
Threshold and Loading invariance
    & \bi[label=$\cdot$] \item Restricts thresholds $\tau$ and loadings $\lambda$ to be equal across groups
                  \item Allows comparison of latent factor variances \ei
    & $\tau_{1,ci} = \tau_{1,c^\prime i}$ for all items, $\forall c, c^\prime$
                      \newline $\tau_{2,ci} = \tau_{2,c^\prime i}$ for non-binary items, $\forall c, c^\prime$
                      \newline $\lambda_{ci} = \lambda_{c^\prime i}$ for all items, $\forall c, c^\prime$
                      \newline For all groups: $\bm{\kappa}=\bm{0}$
                      \newline For ref. group $A$: $\;\begin{array}{|l} \bm{\nu}_A=\bm{0} \\
                                                  \text{diag}(\bm{\Sigma}_A)=\bm{I} \\
                                                  \text{diag}(\bm{\Phi}_A) = \bm{I} \end{array}$ 
                   \\ \\
Threshold, Loading, and Intercept invariance
    & \bi[label=$\cdot$] \item Restricts thresholds $\tau$ and loadings $\lambda$ to be equal across groups
                  \item Restricts intercepts $\nu$ to zero in both groups
                  \item Allows comparison of latent factor variances \emph{and} means \ei
    & $\tau_{1,ci} = \tau_{1,c^\prime i}$ for all items, $\forall c, c^\prime$
                      \newline $\tau_{2,ci} = \tau_{2,c^\prime i}$ for non-binary items, $\forall c, c^\prime$
                      \newline $\lambda_{ci} = \lambda_{c^\prime i}$ for all items, $\forall c, c^\prime$
                      \newline For all groups: $\bm{\nu}=\bm{0}$
                      \newline For ref. group $A$: $\;\begin{array}{|l} \bm{\kappa}_A=\bm{0} \\
                                                  \text{diag}(\bm{\Sigma}_A)=\bm{I} \\
                                                  \text{diag}(\bm{\Phi}_A) = \bm{I} \end{array}$ 
\\ \\ \bottomrule
\end{tabular}
\justify {\scriptsize \emph{Notes}: Adapted from \cite{Wu2016a}.
\par}
\end{table}
