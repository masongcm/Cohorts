\section{Literature \label{sec:lit}}

The importance of cognition in predicting life course success is well established in the economics literature. However, in recent years the role played by `non-cognitive' traits is being increasingly investigated. These traits include constructs as different as psychological and preference parameters such as social and emotional skills, locus of control and self-esteem, personality traits (e.g. conscientiousness), and risk aversion and time preferences. Given the vastity of this literature, we briefly review below the main papers on the determinants and consequences of socio-emotional traits which are more directly related to our work, and we refer to other sources for more exhaustive reviews \citep{borghans2008economics,almlund2011personality,goodman2015social,kautz2014fostering}.

\paragraph{Consequences of socio-emotional traits} One of the first papers to show the importance of non-cognitive personality variables for wages was \citet{bowles2001determinants}. \citet{heckman2006effects} suggested that non-cognitive skills are at least as important as cognitive abilities in determining a variety of adults outcomes. \citet{lindqvist2011labor}, using data based on personal interviews conducted by a psychologist during the Swedish military enlistment exam, show that both cognitive and noncognitive abilities are important in the labour market, but for different outcomes: low noncognitive abilities are more correlated with unemployment or low earnings, while cognitive ability is a stronger predictor of wages for skilled workers. \citet{segal2013misbehavior}, using data on young men from the US National Education Longitudinal Survey, shows that eight-grade misbehaviour is important for earnings over and above eight-grade test scores. \citet{layard2014predicts} find that childhood emotional health (operationalised using the same mother-reported Rutter scale we use in the 1970 British cohort study) at ages 5, 10 and 16 is the most important predictor of adulthood life satisfaction and life course success.

There are only few studies in economics specifically studying ``non-cognitive'' traits and health behaviors. \citet{conti2010education} and \citet{conti2011early} were the first to consider three early endowments, including child socio-emotional traits and health in addition to cognition, using rich data from the 1970 British cohort study. They find strong evidence that non-cognitive traits promote health outcomes and healthy behaviors, and than not accounting for them overestimates the effects of cognition; additionally, they document that child cognitive traits are more important predictors of employment and wages than socio-emotional traits or early health. \citet{chiteji2010time} used the US Panel Study of Income Dynamics (PSID) and found that future orientation and self-efficacy (related to emotional stability) are associated with less alcohol consumption and more exercise. \citet{cobb2014healthy} used the Australian HILDA data and found that an internal locus of control (also related to emotional stability, perceived control over one's life) is related to better health behaviours (diet, exercise, alcohol consumption and smoking). \citet{mendolia2014effect} used the Longitudinal Study of Young People in England and found that individuals with external locus of control, low self-esteem, and low levels of work ethics, are more likely to engage in risky health behaviours. \citet{savelyev2017socioemotional} show that the association between personality traits and health behaviours also holds in a high-IQ sample (the Terman Sample). \citet{heckmanJPE2018} use, instead, early risky behaviours to measure socio-emotional traits, and confirm their predictive power for health behaviours and health outcomes.

Very few papers attempt to make cross-cohorts comparisons about the importance of socio-emotional skills. \citet{blanden2007accounting} -- one of the closes study to ours -- examine cognitive skills, non‐cognitive traits, educational attainment and labour market attachment as mediators of the decline in inter-generational mobility in UK between the 1958 and the 1970 cohorts. The authors take great care in selecting non-cognitive items to be as comparable as possible across cohorts, from the Rutter scale at age 10 for the 1970 cohort and from the Bristol Social Adjustment Guide for the 1958 cohort; however, they do not carry out a formal test of measurement invariance and the do not construct factor scores fully comparable across cohorts as we do. Another paper related to ours is the one by \citet{reardon2016recent}, who study recent trends in income, racial, and ethnic school gaps in several dimensions of school readiness, including academic achievement, self-control, and externalizing behavior, at
kindergarten entry, using comparable data from the Early Childhood Longitudinal Studies (ECLS-K and ECLS-B) for cohorts born from the
early 1990s to the 2000–2010 period in the US. They find that readiness gaps narrowed modestly from 1998 to 2010, particularly between high- and low-income students and between White and Hispanic students. Lastly, \citet{deming2017growing} uses a comparable set of skill measures and covariates across survey waves for the NLSY79 and the NLSY97, and finds that the labour market return to social skills was much greater in the 2000s than in the mid-1980s and 1990s.

\paragraph{Determinants of socio-emotional traits} Equally flourishing has been the literature on the determinants of child socio-emotional skills, which ranges from reduced-form, correlational or causal estimates, to more structural approaches. One of the first papers by \citet{segal2008classroom} has shown that a variety of family and school characteristics predict classroom behaviour. \citet{carneiro2013maternal} study the intergenerational impacts of maternal education, using data from the NLSY79 and an instrumental variable strategy; they find strong effects in terms of reduction in children's behavioural problems. \citet{cunha2010estimating} and \citet{attanasio2018estimating} both estimate production functions for child cognitive and socio-emotional development (in US and Colombia, respectively), and find an important role played by parental investments.

Interventions for improving Social and Emotional Learning (SEL) in a school setting have shown significant improvements in socio-emotional skills, attitudes, behaviours, and academic performance \citep{durlak2011impact}, and a substantial positive return on investments \citep{belfield2015economic}; after-school programs have been shown to be equally effective \citep{durlak2010meta}.

Additionally, it has been shown that a key mechanism through which early childhood interventions improve adult socioeconomic and health outcomes is by boosting socio-emotional skills, such as four teacher-reported behavioural outcomes in the project STAR\footnote{Student's effort, initiative, non-participatory  behavior, and how the student is seen to `value' the class.} \citep{chetty2011does}, reductions in externalising behaviour (from the Pupil Behavior Inventory) at ages 7-9 in the Perry Preschool Project \citep{heckman2013understanding,conti2016effects}, or improvements in task orientation at ages 1-2 in the Abecedarian Project \citep{conti2016effects}.

In sum, even if the literature on the determinants and consequences of socio-emotional skills has been booming, most papers use skills measured in late childhood or in adolescence; and no paper in economics formally tests for invariance of measurements across different groups and constructs fully comparable scores. In this paper, we use measures of child socio-emotional development at age 5, hence before formal schooling starts; and we construct comparable scales across the two cohorts we study (the 1970 and the 2000 British cohorts), so that we can investigate changes in inequality in early development, their determinants, and consequences, in a parallel fashion. 