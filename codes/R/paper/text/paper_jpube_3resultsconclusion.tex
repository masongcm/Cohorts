
%%%%%%%%%%%%%%%%%%%%%%%%%%%%%%%%%%%%%%%%%%%%%%%%%%%%%%%%%%%%%%%%%%%%%%%%%%%%%%%%%%%%%%%%%%%%%%%%%%%%%%%%%

\section{Results}\label{sec:results}

Parameter estimates from our factor model are presented in \autoref{tab:measpars}. As discussed in the previous section, loadings and thresholds are constrained to have the same value across groups. Intercepts are normalised to zero, and error variances to one, for the reference group -- males in the BCS cohort. We use the estimates from this model to predict a score for each child in our sample along the latent externalising and internalising socio-emotional skill dimensions.\footnote{We use an empirical Bayes modal (EBM) approach to estimate the scores. The parameters are estimated using three sources of information. The first is the distribution of the latent variables $\bm{\theta}$, treated as random parameters with a prior $h(\bm{\theta}, \bm{\Omega})$, conditional on the parameters $\bm{\Omega}$. This prior is assumed to be multivariate normal. The second is the observed data $\bm{X}$, and the third is the estimated parameters $\hat{\bm{\Omega}}$. Data and prior are combined into the posterior distribution $w(\bm{\theta}| \bm{X}, \hat{\bm{\Omega}})$. For further details, see Chapter 7 in \citet{Skrondal2004}.} We plot the distribution of the scores in \autoref{fig:dens}. The unit of measurement is standard deviations of the distribution in the subsample of males in the BCS. Given our measurement invariance results in section \ref{sec:measinv}, we stress that the \emph{location} of these scores should not be directly compared across cohorts. However, the shape of the distribution can be given a cross-cohort interpretation. This result is in sharp contrast with what shown by the simple distribution of sum scores in \ref{fig:sumscores}: using raw scores we see an increase in mass only at the top of the distribution, while the factor scores clearly show that there is more mass in both tails of the distribution of the 2000 than of the 1970 cohort.

\subsection{Inequality in socio-emotional skills}

We find that, both unconditionally and for specific groups, inequality in socio-emotional skills at age five has increased between 1975 and 2005/6. \autoref{tab:iqdiff} shows unconditional inequality statistics, using quantile differences in the distribution of skills by gender and cohort. With the exception of internalising skills in female children, all distributions have widened substantially between the BCS and MCS cohorts. The gap for both externalising and internalising skills between the 90th and the 10th percentiles for males has increased by approximately half a standard deviation. The increase in the gap is more pronounced in the bottom half of the distribution. For females, we see a narrowing at the top (90-50), but a widening at the bottom (50-10) of the distribution, again for both externalising and internalising skills.

Inequality has also increased conditional on socioeconomic status. \autoref{fig:ineqps} shows mean skills by maternal education. We compare mothers who continued education with mothers who left school at the minimum compulsory leaving age, according to their year of birth. Given lack of comparability in the level of skills across cohort, we normalise the mean in the `Compulsory' group to zero for both cohorts. For both males and females, and for both externalising and internalising skills, the difference in the socio-emotional skills of their children between more and less educated mothers has increased. The size of the increase is around .1 to .15 of a standard deviation. The increase is particularly pronounced for males, for whom it goes from .20 to .30 for externalising and from .12 to .24 for internalising.

\autoref{fig:ineqsm} shows an even starker pattern when comparing children of mothers who smoked in pregnancy with non-smoking mothers. The fact that maternal smoking during pregnancy is a risk factor for offspring behavioural problems is well known in the medical literature \citep{Gaysina2013}; there is less evidence, however, on whether and to which extent these associations have changed across cohorts. The difference in child skills has increased, from less than .2 to around .4 of a standard deviation, again with the biggest increase experienced by the boys. There is also a significant increase in the gradient by paternal occupation based on social class (\autoref{fig:ineqfsc}), although this is less pronounced if compared to the one based on maternal characteristics. In particular, male children with no father figure living in their household have worse skills (both internalising and externalising) compared to children with blue collar fathers in the MCS cohort. Otherwise, skill differences in father's occupation are mostly constant across the two cohorts.\footnote{Figures \ref{fig:ineqips}, \ref{fig:ineqism}, and \ref{fig:ineqifsc} show inequality in the scale items underlying the factor scores used in this section. The increase in inequality across cohorts is still present, but less marked when looking at these single items. This shows the importance of the factor analysis step in aggregating items, explicitly modelling the measurement error, and testing and accounting for (loadings and thresholds) invariance across the two cohorts.} These patterns are in stark contrast with the findings of \citet{Reardon2016} for the US, who have found a narrowing of the readiness gaps from 1998 to 2010 (however, they have not tested for measurement invariance).

We then examine the same patterns as in the previous figures, but conditional on other family background indicators. The aim is to disentangle the relative contribution of each indicator to socio-emotional skills, and how it has changed in the thirty years between the two cohorts. \autoref{tab:determ} shows coefficients from linear regressions of socio-emotional skills at five on contemporaneous and past socioeconomic indicators, by cohort and gender. Coefficients for indicators in BCS and MCS are presented side by side, together with the $p$-value of the hypothesis that coefficients are the same in the two cohorts.\footnote{We also estimated Tobit models to account for the right truncation of the distribution of skills -- see \autoref{fig:dens}. Tobit estimates are extremely similar to the linear estimates in \autoref{tab:determ}, and are available from the authors upon request.}

Overall, the importance of maternal socioeconomic status (education and in particular employment) in determining socio-emotional skills has increased from the BCS to the MCS children. The `premium' in skills for children of better educated and employed mothers is significantly larger, for both boys and girls, internalising and externalising skills. At the same time, the penalty for having a blue-collar father, or not having a father figure at all in the household, has significantly declined across the two cohorts, especially for girls. Being born to an unmarried mother, and to a mother who smoked during pregnancy, is associated with a higher penalty for both dimensions of socio-emotional skills in the latter cohort.\footnote{It is important to underscore that there has been a significant rise is cohabitation between 1975 and 2006. It is likely that unmarried mothers in the two cohorts have very different characteristics. The choice of this indicator is due to the absence of information on cohabitation in the birth survey for the BCS cohort.} Children of non-white ethnicity have worse internalising and externalising skills in the MCS, a penalty almost absent in the BCS (where the prevalence of non-white children was much lower). Firstborn boys and girls in the BCS have worse skills, but this difference disappears in the MCS. Lastly, we document an increase in the returns to birth weight, which is more pronounced for boys.

These changes in the relative importance of pregnancy factors and family background characteristics for child socio-emotional skills at age 5 need to be interpreted in the light of the significant changes in the prevalence of such characteristics across cohorts. As shown in \autoref{tab:sumstats}, the age of the mother at birth, and the proportion of mothers non-smoking in pregnancy, with post-compulsory education and in employment at the age 5 of the child has substantially increased; at the same time, the proportion of households with no father figure has increased, and so the proportion of women unmarried at birth is much higher in the 2000 than in the 1970 cohort. Also, as noted, the ethnic structure of the population has changed, with a higher proportion of non-white children in the MCS than in the BCS. In general, this has been a period of significant societal changes, with an almost continual rise in the proportion of women in employment, an older age at first birth and a rise in dual-earning parents families \citep{Roantree2018}.

Hence, we lastly attempt to disentangle whether and to which extent the observed changes in inequality in socio-emotional skills across the two cohorts can be attributed to changes in returns (or penalties) to characteristics such as maternal education, or to compositional changes. To this aim, we use the method recently developed by \citet{firpo2018decomposing}\footnote{See also \citet{fortin2011decomposition} for a recent survey of decomposition methods in economics.} as an extension of the Oaxaca-Blinder (OB) decomposition to any distributional measure, that here we apply for the first time to changes in inequality in early childhood development. This two-stage procedure first decomposes distributional changes into a `composition effect' and a `coefficient effect' using a reweighting method; then it further divides these two components into the contribution of each explanatory variable, using Recentered Influence Function (RIF) regression \citep{firpo2009unconditional}.

Following \citet{firpo2018decomposing}, we first perform an OB decomposition using the BCS sample and the counterfactual sample (BCS reweighted to be as MCS)\footnote{We use a logit model to construct the weights and the post-double selection lasso \citep{belloni2012sparse} to select the covariates, among the set of the baseline variables in \autoref{tab:sumstats} and their pairwise interactions.} to get the pure composition effect, using the BCS as reference coefficients. The total unexplained effect in this decomposition corresponds to the specification error, and allows to assess the importance of departures from the linearity assumption. Second, we perform the decomposition using the MCS sample and the counterfactual sample, to obtain the pure coefficient effect (the `unexplained' part); the explained effect in this decomposition corresponds to the reweighting error, which allows to assess the quality of the reweighting.

In \autoref{fig:rif} we present the results of the RIF decomposition for changes in five measures of inequality in socio-emotional skills for the boys, both externalising (top figure) and internalising (bottom figure). The results indicate that different factors explain the rise in inequality in the two skills: on the one hand, compositional changes explain, on average, half of the cross-cohort increase in inequality in externalising skills, regardless of the measure considered;\footnote{The coefficient effects are also sizeable, but imprecisely estimated, with the exception of the variance component.} on the other hand, the increase in inequality in internalising skills seems to be entirely explained (even over-explained) by changes in returns (or penalties) to background characteristics. Composition and coefficient effects are further decomposed in the contribution of each covariate, and the results presented in \autoref{tab:RIF-EXT-M} and in \autoref{tab:RIF-INT-M}. We see in \autoref{tab:RIF-EXT-M} that mother's age and marital status at birth are the two variables that best account for the compositional changes, driving the increase in inequality in externalising skills among the boys for the quantile differences and the variance, respectively. This is hardly surprising, given that we have seen in \autoref{tab:sumstats} that the average age of the mother at birth has increased by approximately three years (from 26 to 29 years old), and that the proportion of unmarried mothers has increased dramatically, from 5\% in the BCS to 36\% in the MCS. The baseline covariates, instead, do a less impressive job at explaining the changes in coefficients underlying the increase in inequality in internalising skills (\autoref{tab:RIF-INT-M}, note the changes in returns to maternal employment go in the direction of reducing inequality). This can be partly explained by the fact that, due to lack of comparable measures across cohorts, we have been unable to account for important determinants of a child's internalising behaviour, such as for example maternal mental health. We also notice that, for the quantile differences 75-25 and 90-50, the composition effect is significant but negative; in other words, compositional changes linked to maternal marriage status would have led to a reduction in inequality, especially at the top of the distribution. Reassuringly, both the specification and the reweighting error are not significantly different from zero. Lastly, the results are not so clear-cut for the girls, who experienced a more muted increase in inequality, concentrated at the bottom of the distribution. The RIF results displayed in \autoref{tab:RIF-F} show that no single contributing factor emerges.

\subsection{Socio-emotional skills and adolescent/adult outcomes}

In this last section, we study the predictive power of socio-emotional skills for adolescent and adult outcomes, to gain some insights as to whether inequality in the early years could translate into later life inequalities. We contribute to a vast interdisciplinary literature by examining medium- and long-term impacts of skills measured at an earlier age than in previous studies, i.e. well before the start of formal education. Showing that these early skills are predictive of different later outcomes across various domains provide a key rationale for the role of early intervention in reducing life course inequalities. In practice, we proceed by regressing health and socioeconomic outcomes measured in adolescence and adulthood on the socio-emotional skills scores at age five obtained by our factor model, controlling for the harmonised family background variables at birth and age five (see \autoref{tab:harmvar}).\footnote{In tables \ref{tab:adol_sum} and \ref{tab:bcs_adultapp}, we show that the conclusions in this section are not sensitive to the factor scoring methodology used. 'Raw' scores, obtained by a simple unweighted average of the item categories in the 11-item subscale have basically equal predictive power to factor scores.} We present results with and without controlling for cognitive skills. As detailed in \autoref{sec:data}, the available cognitive measures are not comparable across cohorts. Still, we control for a factor score that summarises all information on cognitive skills that is available in each cohort, regardless of their comparability.

Socio-emotional skills at five years of age are predictive of adolescent health behaviour and outcomes in both cohorts.\footnote{Unfortunately the strength of the association cannot be directly compared, since the outcomes are measured at different ages: 16 and 14 years for BCS and MCS, respectively.} \autoref{tab:adol} examines adolescent smoking and BMI for both cohorts; \autoref{tab:bcs_adultapp} reports the results for the same outcomes in adulthood (at age 42), for the BCS only. Externalising skills are negatively correlated to subsequent smoking and BMI in both cohorts, for both genders. Recall that a child with high externalising skills exhibits less restless and hyperactive behaviour, and has less anti-social conduct. Our findings are consistent with the body of evidence reviewed in section \ref{sec:lit}, which shows that better socio-emotional skills (measured using different scales and at various points during childhood and adolescence) are negatively associated with smoking. At the same time, internalising skills are positively correlated with smoking (only in the 1970 cohort) and BMI (only for girls), although less strongly than externalising skills. This apparently counterintuitive result makes sense in light of the items in our internalising scale shown in \autoref{tab:scalecomp}. A child with better internalising skills is less solitary, neurotic, and worried. From this perspective, he/she is likely more sociable and subject to peer influence in health behaviours. This is consistent with the evidence in \citet{Goodman2015}, who find a positive association between child emotional health (measured with items from the internalizing behaviour subscale of the Rutter scale at age 10 in the BCS) and smoking at age 42. Furthermore, in recent work \citet{Hsieh2018} have shown personality to be a key mechanism through which peers affect smoking behaviour. We have also tested the robustness of these findings by jointly estimating by maximum likelihood the measurement system (with the partial invariance constraints) and the two outcome equations for smoking and BMI. The results, presented in \autoref{tab:adolML}, are qualitatively similar to those obtained with the two-step method.\footnote{Note that the magnitudes are not exactly comparable because in one-step ML estimation the residual variances of the non-binary measurements also need to be fixed for identification. The remaining parameter estimates are also very similar to those in \autoref{tab:measpars} and available from the authors upon request.}

Conditional on socio-emotional skills, cognition has limited predictive power for these behaviours, and only for girls.\footnote{We do not observe significant associations between early socio-emotional skills and other risky behaviours like drug-taking and alcohol consumption. One possible reason might be the relatively young age at which these skills are measured. Results are available upon request.} This is in line with the evidence in \citet{Conti2010}, who show that not accounting for non-cognitive traits (in their paper, a self-regulation factor measured at age 10) overestimates the importance of cognition for predicting health and health behaviours, using data from the British cohort study. Along the same lines, \citet{Conti2013a} use rich data on child personality and socio-emotional traits collected at ages 7, 11 and 16 in the 1958 British birth cohort,\footnote{They use the Rutter scale and the Bristol Social Adjustment Guide.} and show that these traits rival the importance of cognition in explaining the education gradient in health behaviours (including smoking and BMI). We show that child socio-emotional skills have greater predictive power than cognition for health outcomes and behaviours even when measured at an earlier age than in previous work.

Cohort members from the British Cohort Study are now well into their adulthood. For this cohort, we can examine the association between socio-emotional skills at age five and adult education and labour market outcomes. The structure of \autoref{tab:bcs_adult} is similar to \autoref{tab:adol}, but it considers educational achievement, employment, and earnings (conditional on being in paid employment) for the BCS cohort members. For these outcomes, the predictive power of cognitive skills outweighs that of socio-emotional skills, which are only predictive of educational attainment, and whose predictive power for males is driven to insignificance after controlling for cognition. This is consistent with the evidence in  \citet{Conti2011}, who show that cognitive endowments at age 10 are more predictive (than socio-emotional and health ones) for employment and wage outcomes in the BCS. Again, we show that the greater predictive power of cognition for socioeconomic outcomes holds even when considering earlier-life measures of child development.

\section{Conclusion}\label{sec:conclusions}

In this paper we have studied inequality in a dimension of human capital which has received less attention than others in the literature so far: socio-emotional skills very early in life. In particular, we have focused on the measurements of these skills at age 5 in two British cohorts born 30 years apart: the one of children born in 1970 (British Cohort Study, BCS) and the one of children born in 2000/1 (Millennium Cohort Study, MCS). We have provided a timely contribution to the recent but flourishing literature on the determinants and consequences of early human development, by bridging it with the inequality literature.

We have taken very seriously the issue of comparability of measurements of socio-emotional skills across cohorts. First, we have selected 11 comparable items across two related scales: the Rutter scale in the BCS, and the Strength and Difficulties Questionnaire (SDQ) in the MCS. After examining the latent structure underlying the items, we have identified by means of exploratory factor analysis two dimensions of socio-emotional skills. We have labeled them `internalising' and `externalising' skills, the former related to the ability of children to focus their concentration and the latter to engage in interpersonal activities.

Second, we have formally tested for measurement invariance across the two cohorts (for each gender) of the 11 items comprising the two externalising and internalising scales, following recent methodological advances in factor analysis with categorical outcomes. We have found only partial support for measurement invariance, with the implication that we have only been able to compare how inequality in these socio-emotional skills has changed across the two cohorts, but not whether their average level is higher or lower in one of them. These results sound a warning to research in this area which routinely compares levels of skills across different groups (at different times, or of different gender), without first establishing their comparability.

Third, after having computed comparable scores for both externalising and internalising skills, and for both boys and girls, we have compared how inequality in these skills has changed across the 1970 and the 2000 cohorts. We have documented for the first time that inequality in these early skills has increased, especially for boys. The cross-cohort increase in the gap is more pronounced at the bottom of the distribution (50-10 percentiles). We have also documented changes in conditional skills gaps across cohorts. In particular, the difference in the socio-emotional skills of their children between mothers of higher and lower socio-economic status (education and employment) has increased. The increase in cross-cohort inequality is even starker when comparing children born to mothers who smoked during pregnancy. On the other hand, the skills penalty arising from the lack of a father figure in the household has substantially declined. Moreover, we have formally decomposed the increase in inequality into compositional changes, and changes in returns to maternal characteristics - providing the first child development application of the method recently developed by \citet{firpo2018decomposing}. We have found that half of the increase in inequality in externalising skills across cohorts can be explained by compositional changes, with maternal age and marital status at birth being the most important factors; on the other hand, the increase in inequality in internalising skills seems to be entirely driven by changes in returns to maternal characteristics.

Fourth, we have contributed to the literature on the predictive power of socio-emotional skills by showing that even skills measured at a much earlier age than in previous work are significantly associated with outcomes both in adolescence and adulthood. In particular, socio-emotional skills are more significant predictors of health and health behaviours (smoking and BMI), while cognition has greater predictive power for socioeconomic outcomes (education, employment and wages). Our results ultimately show the importance of inequalities in the early years development for the accumulation of health and human capital across the life course.
