
%%%%%%%%%%%%%%%%%%%%%%%%%%%%%%%%%%%%%%%%%%%%%%%%%%%%%%%%%%%%%%%%%%%%%%%%%%%%%%%%%%%%%%%%%%%%%%%%%%%%%%%%%

\section{Results}\label{sec:results}

Parameter estimates from our factor model are presented in \autoref{tab:measpars}. As discussed in the previous section, loadings and thresholds are constrained to have the same value across groups. Intercepts are normalised to zero, and error variances to one, for the reference group -- males in the BCS cohort. We use the estimates from this model to predict a score for each child in our sample along the latent externalising and internalising socio-emotional skill dimensions.\footnote{We use an empirical Bayes modal (EBM) approach to estimate the scores. The parameters are estimated using three sources of information. The first is the distribution of the latent variables $\bm{\theta}$, treated as random parameters with a prior $h(\bm{\theta}, \bm{\Omega})$, conditional on the parameters $\bm{\Omega}$. This prior is assumed to be multivariate normal. The second is the observed data $\bm{X}$, and the third is the estimated parameters $\hat{\bm{\Omega}}$. Data and prior are combined into the posterior distribution $w(\bm{\theta}| \bm{X}, \hat{\bm{\Omega}})$. For further details, see Chapter 7 in \citet{Skrondal2004}.} We plot the distribution of the scores in \autoref{fig:dens}. The unit of measurement is standard deviations of the distribution in the subsample of males in the BCS. Given our measurement invariance results in section \ref{sec:measinv}, we stress that the \emph{location} of these scores should not be directly compared across cohorts. However, the shape of the distribution can be given a cross-cohort interpretation.\footnote{The density of the scored factors can be contrasted with the distribution of sum scores in \autoref{fig:sumscores}. Using raw scores, instead, shows an increase in mass only at the top of the distribution.} It is immediately visible that there is more mass in the tails of the distribution in the 2000 than in the 1970 cohort.

\subsection{Inequality in socioemotional skills}

We find that, both unconditionally and for specific groups, inequality in socio-emotional skills at age five has increased between 1975 and 2005/6. \autoref{tab:iqdiff} shows unconditional inequality statistics, using quantile differences in the distribution of skills by gender and cohort. With the exception of internalising skills in female children, all distributions have widened substantially between the BCS and MCS cohorts. The gap for both externalising and internalising skills between the 90th and the 10th percentiles for males has increased by approximately half a standard deviation. The increase in the gap is more pronounced in the bottom half of the distribution. For females, we see a narrowing at the top (90-50), but a widening at the bottom (50-10) of the distribution, again for both externalising and internalising skills.

Inequality has also increased conditional on socioeconomic status. \autoref{fig:ineqps} shows mean skills by maternal education. We compare mothers who continued education past the compulsory age with mothers who left school at the compulsory leaving age, according to their year of birth. Given lack of comparability in the level of skills across cohort, we normalise the mean in the `Compulsory' group to zero for both cohorts. For both males and females, and for both externalising and internalising skills, the difference in the socio-emotional skills of their children between more and less educated mothers has increased. The size of the increase is around .1 to .15 of a standard deviation. The increase is particularly pronounced for males, for whom it goes from .20 to .30 for externalising and from .12 to .24 for internalising.

\autoref{fig:ineqsm} shows an even starker pattern when comparing children of mothers who smoked in pregnancy with non-smoking mothers. The fact that maternal smoking during pregnancy is a risk factor for offspring behavioural problems is well known in the medical literature \citep{Gaysina2013}; less evidence there is however, on whether and to which extent these associations have changed across cohorts. The difference in child skills has increased, from less than .2 to around .4 of a standard deviation, again with the biggest increase experienced by the boys. There is also a significant increase in the gradient by paternal occupation based on social class (\autoref{fig:ineqfsc}), although this is less pronounced if compared to the one based on maternal characteristics. In particular, male children with no father figure living in their household have worse skills compared to children with blue collar fathers in the MCS cohort. Otherwise, skill differences in father's occupation are mostly constant across the two cohorts.\footnote{Figures \ref{fig:ineqips}, \ref{fig:ineqism}, and \ref{fig:ineqifsc} show inequality in the scale items underlying the factor scores used in this section. The increase in inequality across cohorts is still present, but less marked when looking at these single items. This shows the importance of the factor analysis step in aggregating items, explicitly modelling the measurement error, and testing and accounting for (loadings and thresholds) invariance across the two cohorts.} These patterns are in stark contrast with the findings of \citet{Reardon2016} for the US, who have found a narrowing of the readiness gaps from 1998 to 2010.

We then examine the same patterns as in the previous figures, but conditional on other family background indicators. The aim is to disentangle the relative contribution of each indicator to socio-emotional skills, and how it has changed in the thirty years between the two cohorts. \autoref{tab:determ} shows coefficients from linear regressions of socio-emotional skills at five on contemporaneous and past socioeconomic indicators, by cohort and gender. Coefficients for indicators in BCS and MCS are presented side by side, together with the $p$-value of the hypothesis that coefficients are the same in the two cohorts.\footnote{We also estimated Tobit models to account for the right truncation of the distribution of skills -- see \autoref{fig:dens}. Tobit estimates are extremely similar to the linear estimates in \autoref{tab:determ}, and are available from the authors upon request.}

Overall, the importance of maternal socioeconomic status (education and in particular employment) in determining socio-emotional skills has increased from the BCS to the MCS children. The `premium' in skills for children of better educated and employed mothers is significantly larger, for both boys and girls, internalising and externalising skills. At the same time, the penalty for having a blue-collar father, or not having a father figure at all in the household, has significantly declined across the two cohorts, especially for girls. Being born to an unmarried mother, and to a mother who smoked during pregnancy, is associated with a higher penalty for both dimensions of socio-emotional skills in the latter cohort, but only for males. This is consistent with recent evidence which shows that family disadvantage disproportionately impedes the pre-market development of boys, in terms of higher disciplinary problems, lower achievement scores, and fewer high-school completions \citep{Autor2016a}. Girls of non-white ethnicity, instead, have worse internalising and externalising skills in the MCS, a penalty not suffered by 5-year old non-white girls in the BCS. Firstborn boys and girls in the BCS have worse skills, but this difference disappears in the MCS. Lastly, we document an increase in the returns to birth weight, which is more pronounced for boys' internalising skills.

These changes in the relative importance of pregnancy factors and family background characteristics for child socio-emotional skills at age 5 need to be interpreted in the light of the significant changes in the prevalence of such characteristics across cohorts. As shown in \autoref{tab:sumstats}, the age of the mother at birth, proportion of mothers non-smoking in pregnancy, with post-compulsory education and in employment at the age 5 of the child has substantially increased; at the same time, the proportion of households with no father figure has increased, and so the proportion of women unmarried at birth is much higher in the 2000 than in the 1970 cohort. Also, the ethnic structure of the population has changed, with a higher proportion of non-white children in the MCS than in the BCS. In general, this has been a period of significant societal changes, with an almost continual rise in the proportion of women in employment, an older age at first birth and a rise in dual-earning parents families \citep{Roantree2018}. However, here we do not attempt to disentangle whether and to which extent the observed changes in inequality in socio-emotional skills across the two cohorts can be attributed to changes in returns (or penalties) to maternal characteristics (such as education and employment) or to compositional changes, like it has been done for the analysis of wage inequality \citep{Blundell2007}.

\subsection{Socio-emotional skills and adolescent/adult outcomes}

In this last section, we study the predictive power of socio-emotional skills for adolescent and adult outcomes. We contribute to a vast interdisciplinary literature by examining medium- and long-term impacts of skills measured at an earlier age than usually examined in previous studies, well before the start of formal education. We do so by regressing health and socioeconomic outcomes measured in adolescence and adulthood on the socio-emotional skills scores at age five obtained by our factor model, controlling for the harmonised family background variables at birth and age five (see \autoref{tab:harmvar}). We present results with and without controlling for cognitive skills. As detailed in \autoref{sec:data}, the available cognitive measures are not comparable across cohorts. Still, we control for a factor score that summarises all information on cognitive skills that is available in each cohort, regardless of their comparability.

Socio-emotional skills at five years of age are predictive of adolescent health behaviour and outcomes in both cohorts.\footnote{Unfortunately the strength of the association cannot be directly compared, since the outcomes are measured at different ages: 16 and 14 years for BCS and MCS, respectively.} \autoref{tab:adol} examines adolescent smoking and BMI for both cohorts; \autoref{tab:bcs_adultapp} reports the results for the same outcomes in adulthood (at age 42), for the BCS only. Externalising skills are negatively correlated to subsequent smoking and BMI in both cohorts, for both genders. Recall that a child with high externalising skills exhibits less restless and hyperactive behaviours, and has less anti-social conduct. Our findings are consistent with the body of evidence reviewed in section \ref{sec:lit}, which shows that better socio-emotional skills (measured using different scales and at various points during childhood and adolescence) are negatively associated with smoking. At the same time, internalising skills are positively correlated with smoking (only in the 1970 cohort) and BMI (only for girls), although less strongly than externalising skills. This apparently counterintuitive result makes sense in light of the items in our internalising scale shown in \autoref{tab:scalecomp}. A child with better internalising skills is less solitary, neurotic, and worried. From this perspective, he is likely more sociable and subject to peer influence in health behaviours. This is consistent with the evidence in \citet{Goodman2015}, who find a positive association between child emotional health (measured with items for the internalizing behaviour subscale of the Rutter scale at age 10 in the BCS) and smoking at age 42. Furthermore, in recent work \citet{Hsieh2018} have shown personality to be a key mechanism through which peers affect smoking behaviour.

Conditional on socio-emotional skills, cognition has limited predictive power for these behaviours, and only for girls.\footnote{We do not observe significant associations between early socio-emotional skills and other risky behaviours like drug-taking and alcohol consumption. One possible reason might be the relatively young age at which these skills are measured. Results are available upon request.} This is in line with the evidence in \citet{Conti2010}, who show that not accounting for non-cognitive traits (a self-regulation factor measured at age 10) overestimates the importance of cognition for predicting health and health behaviours, using data from the British cohort study. \citet{Conti2013a} use rich data on child personality and socio-emotional traits collected at ages 7, 11 and 16 in the 1958 British birth cohort,\footnote{They use the Rutter scale and the Bristol Social Adjustment Guide.} and show that these traits rival the importance of cognition in explaining the education gradient in health behaviours (including smoking and BMI). We show that child socio-emotional skills have greater predictive power for health outcomes and behaviours even when measured at an earlier age.

For the British Cohort Study, we last examine the association between socio-emotional skills at age five and adult education and labour market outcomes. The structure of \autoref{tab:bcs_adult} is similar to \autoref{tab:adol}, but it considers educational achievement, employment, and earnings (conditional on being in paid employment) for the BCS cohort members. For these outcomes, the predictive power of cognitive skills outweighs that of socio-emotional skills, whose predictive power diminishes over time (between the ages 34 and 42), and is driven to insignificance after controlling for cognition. This is consistent with the evidence in \citet{Conti2011}, who show that cognitive endowments at age 10 are more predictive (than socio-emotional and health ones) for employment and wage outcomes in the BCS. Again, we show that the greater predictive power of cognition for socioeconomic outcomes holds even when considering earlier-life measures of child development.

\section{Conclusion}\label{sec:conclusions}

In this paper we have studied inequality in a dimension of human capital which has received limited attention in the literature so far: socio-emotional skills very early in life. In particular, we have focused on the measurements of these skills at age 5 in two British cohorts born 30 years apart: the one of children born in 1970 (British Cohort Study, BCS) and the one of children born in 2000/1 (Millennium Cohort Study, MCS). We have provided several contributions to the recent but flourishing literature on the determinants and consequences of early human development.

We have taken very seriously the issue of comparability of measurements of socio-emotional skills across cohorts. First, we have selected 11 comparable items across two related scales: the Rutter scale in the BCS, and the Strength and Difficulties Questionnaire (SDQ) in the MCS. After examining the latent structure underlying the items, we have identified by means of exploratory factor analysis two dimensions of socio-emotional skills. We have labeled them `internalising' and `externalising' skills, the former related to the ability of children to focus their concentration and the latter to engage in interpersonal activities.

Second, we have formally tested for measurement invariance of the 11 items across the two externalising and internalising scales, following recent methodological advances in factor analysis with categorical outcomes. We have found only partial support for measurement invariance, with the implication that we have only been able to compare how inequality in these socio-emotional skills across the two cohorts has changed, but not whether their average level is larger or smaller in one of the two cohorts. These results sound a warning to research in this area which routinely compares levels of skills across different groups (at different times, or of different gender), without first establishing their comparability.

Third, after having computed comparable scores for both externalising and internalising skills, and for both boys and girls, we have compared how inequality in these skills has changed across the 1970 and the 2000 cohort. We have documented for the first time that inequality in these early skills has increased across cohorts, especially for boys. The cross-cohort increase in the gap is more pronounced at the bottom of the distribution (50-10 percentiles). We have also documented changes in conditional skills gaps across cohorts. In particular, the difference in the socio-emotional skills of their children between mothers of higher and lower socio-economic status (education and employment) has increased. The increase in cross-cohort inequality is even starker when comparing children born to mothers who smoked during pregnancy. In both cases, the increase in inequality is particularly pronounced for boys. On the other hand, the skills penalty arising from the lack of a father figure in the household has substantially declined, especially for girls.

Fourth, we have contributed to the literature on the predictive power of socio-emotional skills by showing that even skills measured at a much earlier age than in previous work are significantly associated with outcomes both in adolescence and adulthood. In particular, socio-emotional skills are more significant predictors of health and health behaviours (smoking and BMI), while cognition has greater predictive power for socioeconomic outcomes (education, employment and wages). Our results show the importance of inequalities in the early years development for the accumulation of health and human capital across the life course. 
