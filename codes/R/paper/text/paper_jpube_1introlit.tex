\section{Introduction}

Human capital is a key determinant of economic growth and performance and of the resources an individual creates and controls over the life cycle. Human capital is also important for various determinants of individual well-being, ranging from health to life satisfaction.  In recent years, the process of human capital accumulation has received considerable attention. There is growing consensus on the fact that human capital is a multidimensional object, with different domains playing different roles in labour market as well as in the determination of other outcomes, including the process of human development. It is also recognised that human capital is the output of a very persistent process, where early years inputs play an important and persistent role.

And yet, there are still large gaps in our knowledge of the process of human capital development. These gaps are partly driven by the scarcity of high quality longitudinal data measuring the evolution over the life cycle of different dimension of human capital. Moreover, there is a lack of consensus on the best measures and on the tools to collect high quality data. As a consequence, even when data are available in different contexts, their comparability is problematic.

In this paper, we focus on an important dimension of human capital, which, so far, has received limited attention: socio-emotional skills. Evidence has shown that gaps in socio-emotional skills emerge at very young ages, and that in the absence of interventions are very persistent across the life cycle \citep{Cunha2006}. However, there is surprisingly little evidence on how inequality in this important dimension of human capital has changed across cohorts. In this paper, we start addressing this gap and focus on the measurement of these skills in two British cohorts: the one of children born in 1970 and the one of children born in 2000. We consider the measurement of socio-emotional skills during early childhood, as these skills have been shown, in a variety of contexts \citep{Almlund2011} to have important long-run effects. Our goal is to characterise the distribution of socio-emotional skills in these cohorts and compare them. In the last part of the paper, we also consider the predictive power of different socio-emotional skills for health and socioeconomic outcomes.

The main contributions of the paper are three. First, we construct a novel scale of of childhood behavioural traits from two validated instruments and assess its comparability across cohorts. By performing an exploratory factor analysis, we determine that two dimensions are a parsimonious representation of socio-emotional skills for both cohorts. Coherently with previous literature, we label them as `internalising' and `externalising' skills, the former relating to the ability of children to focus their drive and determination and the latter relating to their ability in engaging in interpersonal activities. Importantly, for the first time in economics, we study the comparability of the measures in the two cohorts. In particular, we test for \emph{measurement invariance} of the items we use to estimate the latent factors. Intuitively, if one assumes that a set of measures is related to a latent unobserved factor of interest, one can think of this relationship as being driven by the saliency of each measure and the level. If one uses a given measure as the relevant metric for the relevant factor, its saliency will determine the scale of the factor, while some other parameters, which could be driven by the difficulty of a given test or the social norms and attitudes towards a certain type of behaviour, determine the \emph{average level} of the factor. Comparability of estimated factors across different groups (such as different cohorts) assumes  that both the parameters that determine the saliency of a given set of measures and the level of the factors do not vary across groups. We find that, for the measures we use and for both factors, we cannot reject measurement invariance for the saliency parameters. However, we strongly reject measurement invariance for the level parameters. These results imply that while the level of inequality across the two cohorts in the skills we consider is comparable, we cannot determine whether the \emph{average levels} of the two factors are larger or smaller in one of the two cohorts. While this result hinders a comparison in the level of skills, it is of interest per se to find that mothers of English children born in England thirty years apart assess behaviours differently, so that differences in the raw scales cannot be unequivocally interpreted as differences in the underlying skills. We believe this is an important finding which deserves a greater degree of attention in the economic literature.

Third, given the results we obtain on measurement invariance, we proceed to compare the level of inequality in the two types of socio-emotional skills across the two cohorts, for both boys and girls. We find that the most recent cohort is more unequal in both dimensions of socio-emotional skills than the 1970 cohort. This result is particularly apparent for boys, and when looking at differences by maternal background. Fourth, we study whether the socio-emotional skills we observe at a young age are an important determinant of a variety of adolescent (and adult, for the older BCS cohort) outcomes. We find that socio-emotional skills at age five are more predictive than cognitive skills for unhealthy behaviours like smoking and measures of health capital such as body mass index. The effect of cognition, instead, dominates for educational and labour market outcomes.

The rest of the paper is organised as follows. In section \ref{sec:data}, we briefly discuss the data we use in the analysis. In section \ref{sec:methods}, we discuss the methods we use to identify the number of dimensions in socio-emotional skills and how we estimate the latent factors that represent them. In section \ref{sec:measinv}, we discuss the comparability of factors estimated with a given set of measures from different groups and the \emph{measurement invariance} tests we use. Section \ref{sec:results} reports our empirical results, while section \ref{sec:conclusions} concludes the paper.

%%%%%%%%%%%%%%%%%%%%%%%%%%%%%%%%%%%%%%%%%%%%%%%%%%%%%%%%%%%%%%%%%%%%%%%%%%%%%%%%%%%%%%%%%%%%%%%%%%%%%%%%%%%%%%%%%%%%%%
\section{Literature \label{sec:lit}}

The importance of cognition in predicting life course success is well established in the economics literature. However, in recent years the role played by `non-cognitive' traits is being increasingly investigated. These traits include constructs as different as psychological and preference parameters such as social and emotional skills, locus of control and self-esteem, personality traits (e.g. conscientiousness), and risk aversion and time preferences. Given the vastness of this literature, we briefly review below the main papers on the determinants and consequences of socio-emotional traits which are more directly related to our work, and we refer to other sources for more exhaustive reviews \citep{Borghans2008,Almlund2011,Goodman2015,Kautz2014}.

\paragraph{Consequences of socio-emotional traits} One of the first papers to show the importance of non-cognitive personality variables for wages was \cite{Bowles2001}. \cite{Heckman2006} suggested that non-cognitive skills are at least as important as cognitive abilities in determining a variety of adults outcomes. \citet{Lindqvist2011}, using data based on personal interviews conducted by a psychologist during the Swedish military enlistment exam, show that both cognitive and noncognitive abilities are important in the labour market, but for different outcomes: low noncognitive abilities are more correlated with unemployment or low earnings, while cognitive ability is a stronger predictor of wages for skilled workers. \citet{Segal2013}, using data on young men from the US National Education Longitudinal Survey, shows that eight-grade misbehaviour is important for earnings over and above eight-grade test scores. \citet{Layard2014} find that childhood emotional health (operationalised using the same mother-reported Rutter scale we use in the 1970 British cohort study) at ages 5, 10 and 16 is the most important predictor of adulthood life satisfaction and life course success.

There are only few studies in economics specifically studying ``non-cognitive'' traits and health behaviors. \citet{Conti2010a} and \citet{Conti2011} were the first to consider three early endowments, including child socio-emotional traits and health in addition to cognition, using rich data from the 1970 British cohort study. They find strong evidence that non-cognitive traits promote health outcomes and healthy behaviors, and than not accounting for them overestimates the effects of cognition; additionally, they document that child cognitive traits are more important predictors of employment and wages than socio-emotional traits or early health. \citet{Chiteji2010} used the US Panel Study of Income Dynamics (PSID) and found that future orientation and self-efficacy (related to emotional stability) are associated with less alcohol consumption and more exercise. \citet{Cobb-Clark2014} used the Australian HILDA data and found that an internal locus of control (also related to emotional stability, perceived control over one's life) is related to better health behaviours (diet, exercise, alcohol consumption and smoking). \citet{Mendolia2014a} used the Longitudinal Study of Young People in England and found that individuals with external locus of control, low self-esteem, and low levels of work ethics, are more likely to engage in risky health behaviours. \citet{Prevoo2015} construct measures of personality from maternal ratings at 10 and 16 in the British Cohort Study and find that their measure of conscientiousness is positively associated with education and economic outcomes, and negatively associated with body mass index and smoking. \citet{Goodman2015} review the interdisciplinary literature and provide a new analysis of the British Cohort Study, including a particular focus on the role of social and emotional skills (defined using a rich set of measurements of the age 10 sweep) in transmitting `top `job' status between parents and their children. \citet{Savelyev2019} show that the association between personality traits and health behaviours also holds in a high-IQ sample (the Terman Sample). \citet{Heckman2018} use, instead, early risky and reckless behaviours to measure socio-emotional endowments, and confirm their predictive power for education, log wages, smoking and health limits work.

Few papers attempt to make cross-cohorts comparisons about the importance of socio-emotional skills. \citet{Blanden2007} -- one of the closest study to ours -- examine cognitive skills, non‐cognitive traits, educational attainment and labour market attachment as mediators of the decline in inter-generational income mobility in UK between the 1958 and the 1970 cohorts. The authors take great care in selecting non-cognitive items to be as comparable as possible across cohorts, from the Rutter scale at age 10 for the 1970 cohort and from the Bristol Social Adjustment Guide for the 1958 cohort; however, they do not carry out a formal test of measurement invariance and the do not construct factor scores fully comparable across cohorts as we do. Another paper related to ours is the one by \citet{Reardon2016}, who study recent trends in income, racial, and ethnic school gaps in several dimensions of school readiness, including academic achievement, self-control, and externalizing behavior, at kindergarten entry, using comparable data from the Early Childhood Longitudinal Studies (ECLS-K and ECLS-B) for cohorts born from the early 1990s to the 2000–2010 period in the US. They find that readiness gaps narrowed modestly from 1998 to 2010, particularly between high- and low-income students and between White and Hispanic students. \citet{Landerso2017a} study the sources of differences in social mobility between US and Denmark; for the US, they use the antisocial, headstrong, hyperactivtity subscales from the Behavior Problem Index (BPI) in the Children of the NLSY79 (CNLSY), while for Denmark they use orderliness/organization/neatness grades from the Danish written exams.\footnote{As the authors note (footnote 41) ``Our measures of non-cognitive skills in the two countries are clearly not equivalent. The Danish measure of non-cognitive skills is more related to an orderliness/effort measure while the US measure is related to behavioral problems''.} They find that, in both countries, cognitive and non-cognitive skills acquired by age 15 are more important for predicting educational attainment than parental income. Lastly, \citet{Deming2017} uses two sets of skill measures and comparable covariates across survey waves for the NLSY79 and the NLSY97,\footnote{He uses the following four variables as measures of social skills in the NLSY79: self-reported sociability in 1981 and at age 6 (retrospective), the number of clubs in which the respondent participated in high school and participation in high school sports; and the following two variables in the NLSY97: two questions that capture the extroversion factor from the Big 5 Personality Inventory (since measures comparable to the NLSY79 are not available in the NLSY97).} and finds that the labour market return to social skills was much greater in the 2000s than in the mid-1980s and 1990s. \citet{Zilanawala2019} examine differences in socioemotional and cognitive development among 11-year old children in the UK Millennium Cohort Study and the US Early Childhood Longitudinal Survey-Kindergarten Cohort, and find that family resources explain some cross-national differences, however there appears to be a broader range of family background variables in the UK that influence child development. Importantly, none of these papers making comparisons across cohorts or ethnic groups test for measurement invariance like we do.

\paragraph{Determinants of socio-emotional traits} Equally flourishing has been the literature on the determinants of child socio-emotional skills, which ranges from reduced-form, correlational or causal estimates, to more structural approaches. One of the first papers by \citet{Segal2008} has shown that a variety of family and school characteristics predict classroom behaviour. \citet{Carneiro2013} study the intergenerational impacts of maternal education, using data from the NLSY79 and an instrumental variable strategy; they find strong effects in terms of reduction in children's behavioural problems. \citet{Cunha2010} and \citet{Attanasio2018a} both estimate production functions for child cognitive and socio-emotional development (in US and Colombia, respectively), and find an important role played by parental investments. \citet{Moroni2019} estimate production functions for child socio-emotional skills (internalising and externalising behaviour) at age 11 in the UK Millennium Cohort Study, and find that the effects of parental inputs which improve the home environment varies as a function both of the level of the inputs themselves and of the development of the child.

Interventions for improving Social and Emotional Learning (SEL) in a school setting have shown significant improvements in socio-emotional skills, attitudes, behaviours, and academic performance \citep{Durlak2011}, and a substantial positive return on investments \citep{Belfield2015}; after-school programs have been shown to be equally effective \citep{Durlak2010}.

Additionally, it has been shown that a key mechanism through which early childhood interventions improve adult socioeconomic and health outcomes is by boosting socio-emotional skills, such as four teacher-reported behavioural outcomes in the project STAR\footnote{Student's effort, initiative, non-participatory  behavior, and how the student is seen to `value' the class.} \citep{Chetty2011a}, reductions in externalising behaviour (from the Pupil Behavior Inventory) at ages 7-9 in the Perry Preschool Project \citep{Heckman2013,Conti2016a}, or improvements in task orientation at ages 1-2 in the Abecedarian Project \citep{Conti2016a}.

In sum, even if the literature on the determinants and consequences of socio-emotional skills has been booming, most papers use skills measured in late childhood or in adolescence; and no paper in economics formally tests for invariance of measurements across different groups and constructs fully comparable scores. In this paper, we use measures of child socio-emotional development at age 5, hence at the very beginning of formal schooling; and we construct comparable scales across the two cohorts we study (the 1970 and the 2000 British cohorts), so that we can investigate changes in inequality in early development, their determinants, and consequences, in a parallel fashion. 
