\section{Introduction}

Human capital is a key determinant of economic growth and performance and of the resources an individual creates and controls over the life cycle. Human capital is also important for various determinants of individual well-being, ranging from health to life satisfaction.  In recent years, the process of human capital accumulation has received considerable attention. There is growing consensus on the fact that human capital is a multidimensional object, with different domains playing different roles in labour market as well as in the determination of other outcomes, including the process of human development. It is also recognised that human capital is the output of a very persistent process, where early years inputs play an important and persistent role.

And yet, there are still large gaps in our knowledge of the process of human capital development. These gaps are partly driven by the scarcity of high quality longitudinal data measuring the evolution over the life cycle of different dimension of human capital. Moreover, there is a lack of consensus on the best measures and on the tools to collect high quality data. As a consequence, even when data are available in different contexts, their comparability is problematic.

In this paper, we focus on an important dimension of human capital, which, so far, has received limited attention: socio-emotional skills. Evidence has shown that gaps in socio-emotional skills emerge at very young ages, and that in the absence of interventions are very persistent across the life cycle \citep{cunha2006interpreting}. However, there is surprisingly little evidence on how inequality in this important dimension of human capital has changed across cohorts. In this paper, we start addressing this gap and focus on the measurement of these skills in two British cohorts: the one of children born in 1970 and the one of children born in 2000. We consider the measurement of socio-emotional skills during early childhood, as these skills have been shown, in a variety of contexts \citep{almlund2011personality} to have important long-run effects. Our goal is to characterise the distribution of socio-emotional skills in these cohorts and compare them. In the last part of the paper, we also consider the predictive power of different socio-emotional skills for health and socioeconomic outcomes.

The main contributions of the paper are four. First, we use two validated scales of childhood behavioural traits and select those items which are comparable across the two cohorts. By performing an exploratory factor analysis, we determine that we need at least two dimensions to characterise socio-emotional skills. We label them as `internalising' and `externalising' skills, the former relating to the ability of children to focus their drive and determination and the latter relating to their ability in engaging in interpersonal activities.

Second, we study the comparability of the measures in the two cohorts. In particular, we test for \emph{measurement invariance} of the items we use to estimate the latent factors. Intuitively, if one assumes that a set measures is related to a latent unobserved factor of interest, one can think of this relationship as being driven by the saliency of each measure and the level. If one uses a given measure as the relevant metric for the relevant factor, its saliency will determine the scale of the factor, while some other parameters, which could be driven by the difficulty of a given test or the social norms and attitudes towards a certain type of behaviour, determine the \emph{average level} of the factor. Comparability of estimated factors across different groups (such as different cohorts) assumes  that both the parameters that determine the saliency of a given set of measures and the level of the factors do not vary across groups. We find that, for the measures we use and for both factors, we cannot reject measurement invariance for the saliency parameters. However, we strongly reject measurement invariance for the level parameters. These results imply that while the level of inequality across the two cohorts in the skills we consider is comparable, we cannot determine whether the \emph{average levels} of the two factors are larger or smaller in one of the two cohorts.

Third, given the results we obtain on measurement invariance, we proceed to compare the level of inequality in the two types of socio-emotional skills across the two cohorts, for both boys and girls. We find that the most recent cohort is more unequal in both dimensions of socio-emotional skills than the 1970 cohort. This result is particularly apparent for boys, and when looking at differences by maternal background. Fourth, we study whether the socio-emotional skills we observe at a young age are an important determinant of a variety of adolescent (and adult, for the older BCS cohort) outcomes. We find that socio-emotional skills at age five are more predictive than cognitive skills for unhealthy behaviours like smoking and measures of health capital such as body mass index. The effect of cognition, instead, dominates for educational and labour market outcomes.

The rest of the paper is organised as follows. In section \ref{sec:data}, we briefly discuss the data we use in the analysis. In section \ref{sec:methods}, we discuss the methods we use to identify the number of dimensions in socio-emotional skills and how we estimate the latent factors that represent them. In section \ref{sec:measinv}, we discuss the comparability of factors estimated with a given set of measures from different groups and the \emph{measurement invariance} tests we use. Section \ref{sec:results} reports our empirical results, while section \ref{sec:conclusions} concludes the paper.
